\documentclass[11pt]{article}

\author{Vassilios Kaxiras}
\title{Physics 160 Pset 1}

\usepackage{graphicx}
\usepackage{amsthm}
\usepackage{amsmath}
\usepackage{amsfonts}
\usepackage[bottom]{footmisc}
\usepackage[margin=1in]{geometry}
\usepackage[shortlabels]{enumitem}
\usepackage{graphicx}
\usepackage{comment}
\usepackage{float}
\usepackage{tikz}
\usepackage{amssymb}

\theoremstyle{definition}
\newtheorem{definition}{Definition}
\newtheorem{theorem}{Theorem}
\newtheorem{principle}{Principle}
\newtheorem*{solution}{Solution}

\usepackage{mdframed}
\newcommand{\Mod}[1]{\,(\mathrm{mod}\,#1)}

\renewcommand{\u}[1]{\underline{#1}}

\newcommand{\eq}{\begin{equation}\begin{aligned}}
\newcommand{\qe}{\end{aligned}\end{equation}}
\newcommand{\bra}[1]{\langle #1|}
\newcommand{\ket}[1]{|#1\rangle}
\newcommand{\Bra}[1]{\left\langle #1\right|}
\newcommand{\Ket}[1]{\left|#1\right\rangle}
\newcommand{\braket}[2]{\langle #1|#2\rangle}
\newcommand{\mat}{\begin{bmatrix}}
\newcommand{\tam}{\end{bmatrix}}
\newcommand{\tr}{\text{Tr}}

\newcommand{\img}[4][0.5\textwidth]{
  \begin{figure}[h]
    \centering
    \includegraphics[width=#1]{#2}
    \caption{#3}
    \label{#4}
  \end{figure}
}

\newenvironment{coherence}{\begin{mdframed}\begin{center}\underline{Coherence Checks}\end{center}}{\end{mdframed}}

\usepackage{fancyhdr}
\pagestyle{fancy}
\lhead{Vassilios Kaxiras}

\begin{document}

\maketitle

\begin{solution}[1]
  Consider a 3-bit gate $G$ with inputs $A,B,C$ and outputs $A',B',C'$.
  \begin{center}
    \begin{tabular}{ccc|ccc}
      A & B & C & A' & B' & C' \\
      \hline
      0 & 0 & 0 & 1 & 0 & 0 \\
      0 & 1 & 0 & 1 & 1 & 0 \\
      0 & 0 & 1 & 1 & 0 & 1 \\
      0 & 1 & 1 & 0 & 1 & 1 \\
      1 & 0 & 0 & 0 & 0 & 0 \\
      1 & 1 & 0 & 0 & 1 & 0 \\
      1 & 0 & 1 & 0 & 0 & 1 \\
      1 & 1 & 1 & 1 & 1 & 1
    \end{tabular}
  \end{center}
  This gate is reversible, since all the outputs are simply a permutation of all the inputs. And, we can get NAND from this:
  \eq
    \text{NAND}(x,y)=G(0,x,y)_{(1)}
  \qe
  where $G_{(1)}=A'$.
\end{solution}

\begin{solution}[2]
  \begin{enumerate}[(a)]
    \item 
    \begin{enumerate}[i.]
      \item Yes
      \eq
        C^{\dagger}=(A+B)^{\dagger}=A^{\dagger}+B^{\dagger}=A+B=C
      \qe
      \item No, unless $A,B$ commute.
      \eq 
        C^{\dagger}=(AB)^{\dagger}=B^{\dagger}A^{\dagger}=BA
      \qe
      which is not necessarily equal to $AB=C$.
      \item No, unless $C=0$.
      \eq
        C^{\dagger}=(i(A+B))^{\dagger}=-i(A+B)^{\dagger}=-i(A^{\dagger}+B^{\dagger})=-i(A+B)=-C
      \qe
      \item Yes
      \eq
        C^{\dagger}=(AB+BA)^{\dagger}=B^{\dagger}A^{\dagger}+A^{\dagger}B^{\dagger}=BA+AB=C
      \qe
      \item Yes
      \eq
        C^{\dagger}=(i(AB-BA))^{\dagger}=-i(B^{\dagger}A^{\dagger}-A^{\dagger}B^{\dagger})=i(AB-BA)=C
      \qe
    \end{enumerate}
    \item TODO
    \item 
  \end{enumerate}
\end{solution}

\end{document}
